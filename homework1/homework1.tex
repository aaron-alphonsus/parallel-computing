\documentclass{article}
\usepackage[utf8]{inputenc}

\usepackage[a4paper, top=3cm]{geometry} % Reduce top margin
\usepackage{listings}
\usepackage{enumitem} 

\title{Homework 1\\ CSC410 - Parallel Computing}
\author{Aaron G. Alphonsus}
\date{\today}

\begin{document}
\maketitle

% 81 column comment
%01234567890123456789012345678901234567890123456789012345678901234567890123456789

\begin{enumerate}

\item The identity values for each operator is as follows: 
\begin{table}[ht]
    \setlength{\parindent}{10ex}
    \begin{tabular}{ll}
        \&\&                 & 1 \\
        $||$                 & 0 \\
        $|$                  & 0 \\
        \textasciicircum{}   & 0
    \end{tabular}
\end{table}

\item We can do this by declaring a private variable inside the 
\texttt{parallel} block and initializing it with the identity value of the 
operator we are using. We then use the \texttt{critical} pragma before updating 
the global variable.

\item 
\begin{enumerate}[label=(\alph*)]

    \item 
\begin{lstlisting}
# pragma omp parallel for num_threads(thread_count) \
    default(none) private(i) shared (a, b, x)
    for(i=0; i<(int) sqrt(x); i++) {
       a[i] = 2.3*i;
       if (i < 10) b[i] = a[i];
    }
\end{lstlisting}

    \item 
    This seems to be a sequential program. It looks like the objective is
    to start from the beginning of the \texttt{a} array and set each value to
    \texttt{2.3*i}. The program should stop after the first instance of
    \texttt{a[i] < b[i]} however, if we parallelize this, the flag could be set
    at the wrong time and cause an early loop termination.

    \item
% should foo be included?
\begin{lstlisting}
# pragma omp parallel for num_threads(thread_count) \
    default(none) private(i) shared (a, n) 
    for(i=0; i<n; i++)
       a[i] = foo(i);
\end{lstlisting}

    \item
\begin{lstlisting}
# pragma omp parallel for num_threads(thread_count) \
    default(none) private(i) shared (a, b, n)
    for(i=0; i<n; i++) {
        a[i] = foo(i);
        if(a[i] < b[i]) a[i] = b[i];
    }
\end{lstlisting}

    \item
    Similar to 3(b), this is not suitable for parallel execution because it
    could cause an early loop termination when one of the threads executes the 
    \texttt{break} statement.
    
    \item
    % need to include private, shared?
\begin{lstlisting}
dotp = 0;
# pragma omp parallel for num_threads(thread_count) \
    default(none) private(i) shared (dotp, a, b, n) \
    reduction(+: dotp)
    for(i=0; i<n; i++)
        dotp += a[i]*b[i];
\end{lstlisting}
        
    \item
\begin{lstlisting}
# pragma omp parallel for num_threads(thread_count) \
    default(none) private(i) shared (a, k) 
    for(i=k; i<2*k; i++)
        a[i] = a[i] + a[i-k];
\end{lstlisting}

    \item
    This is similar to problem 3(g) but we need to be careful. Groups of
    \texttt{k} positions may be filled in parallel but each subsequent group of
    \texttt{k} positions depend on the preceding group
    
\end{enumerate} 

\item
Given that this is an \texttt{m}-stage pipeline and the task has \texttt{m} 
sub-tasks, when \texttt{Task 1} comes in, it is completed after \texttt{m} 
cycles. However, as the pipeline has now filled up, every cycle that follows 
completes another task. So the completion times of \texttt{Task 2}, 
\texttt{Task 3}, and \texttt{Task 4} are \texttt{m+1}, \texttt{m+2}, and 
\texttt{m+3} respectively. 

Therefore, for an \texttt{m}-stage pipeline executing \texttt{m} sub-tasks that 
each require 1 unit of time, \texttt{n} tasks can be processed in \texttt{m+n-1}
time.

\item
If the address of the nodes in a hypercube has \textit{n} bits, it can have 
$2^n$ nodes at most and each node will have \textit{n} edges.

Perform an \texttt{XOR} between \texttt{u} and \texttt{v}. Count all the 1
bits. This gives you the minimum number of steps to reach \texttt{v}. You may
start from either end and flip each bit if they are different, keep them same if
they match. 

\begin{lstlisting}
for each bit of u
    if it differs from the bit of v in the same position 
        change the bit // i.e. move to the neighboring node
    // else stay at the same node
	
e.g. u = 110, v = 011 
	Move to: 111
	Move to: 111 
	Move to: 011
\end{lstlisting}

\end{enumerate} 


\end{document}